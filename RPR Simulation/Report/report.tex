\documentclass[letterpaper,12pt]{report}
\usepackage{pstricks}
\usepackage{amsmath}
\usepackage{wrapfig}

\begin{document}


% Title
\title{ Force Control of an RPR manipulator\\EL 522 Final Project}
\author{Griswald Brooks}
\maketitle
 
\begin{abstract}
While positioning of manipulator end effectors is an important task, often applying a particular force to the environment is needed.
This paper describes a basic hybrid force control technique and presents simulation results.
\end{abstract}

\tableofcontents

\chapter{Introduction}

\section{Concept and Motivation}
Many manipulator applications require specific amounts of force to be applied to the environment. 
Tasks include machining, object manipulation, and part assembly. 
Without force control, the robot would apply an arbitrary amount of force as a function 
of its joint error and interaction with the environment. This would lead to unpredictable
results and possible manipulator instability, e.g. causing the end effector to bounce.

One method of force control is a feed forward method called impedance control. In this method 
the environment is modeled as a mass-spring or a mass-spring-dampened system 
and the required displacement needed for a desired force is calculated \eqref{eq:displacementeq1}. 
The manipulator is then commanded to its intended tracking plus this displacement. This can be though of
as the manipulator tracking an inner virtual wall.


\begin{equation} \label{eq:displacementeq1}
\delta x = {F_d \over k}
\end{equation}

\begin{figure}
\centering
% Spring
\def\zigzag{\psline(0,0)(0.5,0.5)(1.5,-0.5)(2,0)}
\psset{unit=0.25,linewidth=1.5pt}
\multips(-1,0)(2,0){4}{\zigzag}
\psline[linewidth=1.5pt](-2,0)(-1,0)
\psline[linewidth=1.5pt](8,0)(7,0)
% Spring label
\rput(4.5,2.5){$k$}
% Virtual Wall
\psline[linewidth=1.5pt,linestyle=dashed, dash=3pt 3pt](1,-3)(1,3)
% Ground
\pspolygon[linecolor=white,fillstyle=hlines](8,-3)(8,3)(8.9,3)(8.9,-3)
\psline[linewidth=1.5pt](8,-3)(8,3)
% Wall
\psline[linewidth=1.5pt](-2,3)(-2,-3)
% Force arrow
\psline[linewidth=3pt]{->}(-5,2)(1,2)
% Force label
\rput(-7,2){$F_d$}
% dx arrow
\psline[linewidth=1pt]{<->}(-2,-1.75)(1,-1.75)

% dx label
\rput(-.5,-1){$\delta x$}

\caption{Mass-spring wall model with virtual wall.}
\end{figure}


Where $F_d$ is the desired force, $k$ is the modeled wall spring constant, and $\delta x$ is the wall displacement necessary to produce $F_d$.
One drawback of this method is that it assumes perfect knowledge of the environment. If the wall stiffness varies the applied force will vary.
To supplement this, a force/torque sensor can be added to the end effector which provides feedback to the controller allowing the applied force
to be modulated. This method is known as Hybrid Force Control and is the technique used in this report.

\section{Manipulator Model}

The manipulator considered in this project is the RPR manipulator. It has a revolute joint at its base and wrist which are coplanar, 
and a prismatic joint in between, Figure \ref{fig:RPRfig}. 

\begin{figure}[t]
\centering
% link 1&2
\pspolygon[linewidth=1.5pt](-2.0884,0.3384)(-1.9116,0.1616)(0.2097,2.2829)(0.0329,2.4597)
% link 3
\pspolygon[linewidth=1.5pt](0.1213,2.4963)(0.1213,2.2463)(3,2.2463)(3,2.4963)
% base block
\pspolygon[linewidth=1.5pt](-1.75,-0.25)(-2.25,-0.25)(-2.25,0.25)(-1.75,0.25)
% joint 2
\pspolygon[linewidth=1.5pt,fillstyle=solid](-0.5858,2.0178)(-0.2322,1.6642)(-1.8232,0.0732)(-2.1768,0.4268)
\pspolygon[linewidth=1.5pt,fillstyle=solid](-0.8156,1.6112)(-0.6388,1.4344)(-1.0631,1.0101)(-1.2399,1.1869)
\pspolygon[linewidth=1.5pt,fillstyle=solid,fillcolor=black](-1.2399,1.1869)(-1.0631,1.0101)(-1.2752,0.7980)(-1.4520,0.9748)
% joint 1
\pscircle[linewidth=1.5pt,fillstyle=solid](-2,0.25){0.25}
\pscircle[linewidth=1.5pt,fillstyle=solid,fillcolor=black](-2,0.25){0.1}
% joint 3
\pscircle[linewidth=1.5pt,fillstyle=solid](0.1213,2.3713){0.25}
\pscircle[linewidth=1.5pt,fillstyle=solid,fillcolor=black](0.1213,2.3713){0.1}
% manipulator
\pspolygon[linewidth=1.5pt](3,2.7463)(3,1.9963)(3.5,1.9963)(3.5,2.2463)(3.25,2.2463)(3.25,2.4963)(3.5,2.4963)(3.5,2.7463)
\caption{Schematic diagram of RPR manipulator}
\label{fig:RPRfig}
\end{figure}

\subsection{Kinematics}
The kinematic model of the manipulator is given by the transformation matrices
\eqref{eq:Amatrices}, resulting in an end effector transformation of \eqref{eq:A1A2A3}. The DH parameters for the 
manipulator are given in Table \ref{tab:DHtable}.

\begin{subequations}
		\begin{equation}
		A_1 = 
		\begin{bmatrix}
			cos\theta_1&0&sin\theta_1&0\\
			sin\theta_1&0&-cos\theta_1&0\\
			0&1&0&0\\
			0&0&0&1\\
		\end{bmatrix}
		\end{equation}
		\begin{equation}
		A_2 = 
		\begin{bmatrix}
			1&0&0&0\\
			0&0&1&0\\
			0&-1&0&d_2\\
			0&0&0&1\\
		\end{bmatrix}
		\end{equation}
		\begin{equation}
		A_3 = 
		\begin{bmatrix}
			cos\theta_3&-sin\theta_3&0&a_3cos\theta_3\\
			sin\theta_3&cos\theta_3&0&a_3sin\theta_3\\
			0&0&1&0\\
			0&0&0&1
		\end{bmatrix}
		\end{equation}
\label{eq:Amatrices}
\end{subequations}
\begin{equation}
A_1A_2A_3 = 
		\begin{bmatrix}
			c_{13}&-s_{13}&0&a_3c_{13}+d_2s_1\\
			s_{13}&c_{13}&0&a_3s_{13}-d_2c_1\\
			0&0&1&0\\
			0&0&0&1
		\end{bmatrix}
\label{eq:A1A2A3}
\end{equation}
\begin{table}
\centering
	\begin{tabular}{r|c c c c}
		Joint&$\alpha_i$&$a_i$&$d_i$&$\theta_i$\\
		\hline
		1&0&$90^\circ$&0&$\theta_1$\\
		2&0&$-90^\circ$&$d_2$&0\\
		3&$a_3$&0&0&$\theta_3$
	\end{tabular}
\caption{DH parameters for RPR manipulator.}
\label{tab:DHtable}
\end{table}

Using the standard input tranformation matrix seen in \eqref{eq:Hin},the inverse kinematics
of the manipulator are given by \eqref{eq:IK}. These were used to calculate the desired joint positions,
given an end effector reference trajectory.
\begin{equation}
H_{in} = 
	\begin{bmatrix}
		r_{11}&r_{12}&r_{13}&d_x\\
		r_{21}&r_{22}&r_{23}&d_y\\
		r_{31}&r_{32}&r_{33}&d_z\\
		0&0&0&1
	\end{bmatrix}
\label{eq:Hin}
\end{equation}
\begin{subequations}
	\begin{align}
		\theta_1 + \theta_3 &= atan2(r_{21},r_{11})\\
		\theta_1 &= atan2(d_x-a_3c_{13},a_3s_{13}-d_y)\\
		d_2 &= \sqrt{(d_x-a_3c_{13})^2+(d_y-a_3s_{13})^2}\\
		\theta_3 &= atan2(r_{21},r_{11}) - \theta_1
	\end{align}
\label{eq:IK}
\end{subequations}
The desired joint velocities $\dot q$ were given using the inverse Jacobian, \eqref{eq:IJ}.
This involved first calculating the Jacobian, \eqref{eq:Jacob}, and numerically calculating
the Moore-Penrose pseudoinverse.
\begin{equation}
\dot q = J^\#v
\label{eq:IJ}
\end{equation}
Where $v$ is a $6\times1$ column vector containing the three end effector translational velocities and three angular
velocities.
\begin{equation} \label{eq:Jacob}
J = 
	\begin{bmatrix}
		-a_3s_{13}+d_2c_1&s_1&-a_3s_{13}\\
		a_3c_{13}+d_2s_1&-c_1&a_3c_{13}\\
		0&0&0\\
		0&0&0\\
		0&0&0\\
		1&0&1
	\end{bmatrix}
\end{equation}

Following this, the desired joint accelerations $\ddot q$ were given by \eqref{eq:IA} and calculated
using the aforementioned inverse Jacobian and the element-wise time derivative
of the Jacobian \eqref{eq:Jdot}.
\begin{equation} \label{eq:Jdot}
\dot J = 
	\begin{bmatrix}
		-a_3c_{13}(\dot\theta_1+\dot\theta_3)+\dot d_2c_1-d_2s_1\dot\theta_1&c_1\dot\theta_1&-a_3c_{13}(\dot\theta_1+\dot\theta_3)\\
		-a_3s_{13}(\dot\theta_1+\dot\theta_3)+\dot d_2s_1+d_2c_1\dot\theta_1&s_1\dot\theta_1&-a_3s_{13}(\dot\theta_1+\dot\theta_3)\\
		0&0&0\\
		0&0&0\\
		0&0&0\\
		0&0&0
	\end{bmatrix}
\end{equation}
\begin{equation} \label{eq:IA}
\ddot q = J^\#(\dot v - \dot J \dot q)
\end{equation}

\subsection{Dynamics}
Modeling the manipulator as a collection of point masses located at the revolute joints and end effector,
the dynamics of the manipulator were given by \eqref{eq:dyn}.
\begin{equation} \label{eq:dyn}
D(q)\ddot q + C(q,\dot q) + G(q) = U + B(q) + J(q)^TF
\end{equation}

\section{Environmental Model}
\section{Feedback Linearization}
\section{Hybrid Force Control}

\chapter{Results}
\section{Position Error}
\section{Force Error}


\chapter{Conclusion}


\end{document}